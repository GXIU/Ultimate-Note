\chapter{分工的形成}

分工是城市理论中的重要一环。近期的研究工作有:

\begin{itemize}
    \item  The Spatial Division of Talent in City Regions: Location Dynamics of Business Services in Copenhagen
\end{itemize}

\section{Fitness}

\href{https://www.nature.com/articles/s41586-018-0422-6}{Fitness benefits and emergent division of labour at the onset of group living}

群体生活的初始fitness的优势通常被认为是社会性演化的桎梏(要有社会性演化,先要有一点点初始的fitness)。进化理论预测这种优势需要在很小的社群中就出现。通常认为,这种优势与群体效率的标度律相关。在社会性昆虫和其他语境下,群体生活的优势被放在了相对于DOL的核心位置(DOL:个体之间的差异性,个体内部的一致性)。但是,社会性群体的“入侵”很可能很微弱,而且对于互补功能来说,相同个体数量可能是过剩的。自组织理论指出,DOL可以在很小、很简单的群体中生成。但是,关于群体规模效应作用于的DOL经验数据依然是模糊的。

本文使用长时间序列的自动化行为追踪,结合数学建模的方式,证明社会种群规模的变化可以在极度相似的,小到只有6个workers的群体中产生DOL。这些早期行为与稳态(homeostasis,保持colony和个体fitness的稳定条件)的大幅提升有关系。我们的模型指出,稳态的提升主要又与群体规模扩大相关,次要与更高度的DOL相关。我们的结果指出:DOL、稳态性提高、更高的fitness三者可以在小型同质的自然的社会族群中出现。因此,与提升社会群体规模相关的标度律才可以提升初始阶段群体生活的社会聚合力(cohesion)。