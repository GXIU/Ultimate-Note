\chapter{分工的形成}

分工是城市理论中的重要一环。近期的研究工作有:

\begin{itemize}
    \item  The Spatial Division of Talent in City Regions: Location Dynamics of Business Services in Copenhagen
\end{itemize}

\section{Fitness}

\href{https://www.nature.com/articles/s41586-018-0422-6}{Fitness benefits and emergent division of labour at the onset of group living}

群体生活的初始fitness的优势通常被认为是社会性演化的桎梏(要有社会性演化,先要有一点点初始的fitness)。进化理论预测这种优势需要在很小的社群中就出现。通常认为,这种优势与群体效率的标度律相关。在社会性昆虫和其他语境下,群体生活的优势被放在了相对于DOL的核心位置(DOL:个体之间的差异性,个体内部的一致性)。但是,社会性群体的“入侵”很可能很微弱,

However, at the onset of sociality groups were probably small and composed of similar individuals with potentially redundant—rather than complementary—function1. Self-organization theory suggests that division of labour can emerge even in relatively small, simple groups9,10. However, empirical data on the effects of group size on division of labour and on fitness remain equivocal6. Here we use long-term automated behavioural tracking in clonal ant colonies, combined with mathematical modelling, to show that increases in the size of social groups can generate division of labour among extremely similar workers, in groups as small as six individuals. These early effects on behaviour were associated with large increases in homeostasis—the maintenance of stable conditions in the colony11—and per capita fitness. Our model suggests that increases in homeostasis are primarily driven by increases in group size itself, and to a smaller extent by a higher division of labour. Our results indicate that division of labour, increased homeostasis and higher fitness can emerge naturally in social groups that are small and homogeneous, and that scaling effects associated with increasing group size can thus promote social cohesion at the incipient stages of group living.