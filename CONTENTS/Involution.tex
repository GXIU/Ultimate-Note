\chapter{内卷化}

内卷化是我最近看到的一个高频词。偶然间想到这个概念与我们平时讲城市体系的层次结构似乎有着很好的对应关系,就决定在笔记里开一章做一些思考。

\section{介数中心性}

指标定义我们从一些已有的文献开始看。比如介数中心度的这篇~\cite{Barthelemy2018BC}

\section{进化动力学}

\section{Growth models}

生长模型是处理这类问题的一个重要出发点。下面是之前的一个文献笔记。

\subsection{Protocol:spatial-constrained attachment}
\begin{enumerate}
\item
  给定一个有界\(d\)维欧式空间\(S=|x_1|,\cdots,|x_d|\le\frac{L}{2}.\)
  \(t=0\)时在原点插入一个结点.
\item
  结点生成:每个\(t\)时刻,以均匀分布在\(S\)中放置一个新结点\(P_t\).
  记它的坐标为\(x_t.\) 如果存在一个结点\(P_q,q\in\{1,2,\cdots,t-1\}\),
  使得\(||x_p-x_q||< r\), 则结点\(P_t\)存活。否则\(P_t\)死亡。
\item
  连边:将新加入结点与其\(r-\)临域的所有结点相连。
\item
  重复这个过程,直到存活的结点达到\(N\)个。
\end{enumerate}

这个几何网络会加速增长。因为新加入的结点可生存的区域的测度(为所有存活结点的\(r-\)临域的开覆盖)会越来越大。

\subsection{解析}
\begin{itemize}
\item
  所需要的极限:\textbf{热力学极限}:\(L,t\rightarrow\infty.\)

  在这种情况下,网络会渐近形成一个\(d\)维球。
\item
  半径\(R(t)\)(定义为\(\max\{||P_0-P_i||,i=1,2,\cdots t\}\))

  \begin{itemize}
  \item
    有\(R(t)\sim t.\)
  \item
    证明:以1维正半轴为例,新结点\(P(t+1)\)落在\([0,R(t)+r)\)可以生存,落在\([R(t)-r,R(t)+r]\)可以使得\(R(t+1)> R(t).\)
    于是增加半径的量的期望为(在\(R(t)\)远大于\(r\)的前提下)

    \begin{align}
    &\int_0^Ldx\ P(x_{t+1}=x)\Delta R_{x,t}\\
    =& \frac{2r}{L/2}\int_{R(t)-r}^{R(t)+r} dx\ \frac{1}{2r} (x+r-R(t))\\
    =& \frac{2}{L} \{(2r\cdot [r-R(t)]) +\frac{1}{2}[(R(t)+r)^2-(R(t-r))^2] \}\\
    =& \frac{2}{L} 2r^2\\
    =&r^2/L\\\
    \equiv& C 
    \end{align}

    所以网络的半径\(R(t)\)匀速增长。所以\(R(t)\sim t. \)
    \(\text{q.e.d.}\)

    对于多维情形,在每个维度上的半径\(R_1,\cdots,R_d\)的增长速率都相同,所以它们的线性组合的增长速率相同。即向任何一个方向的增长速率相同。证毕。
  \end{itemize}
\item
  距中心距离为\(\rho\)时,密度\(\mu(\rho,\Theta ,t)\sim \frac{R(t)-\rho}{L^d},\)
  \(\mu(\rho,t) \sim \frac{(R(t)-\rho)\rho^{d-1}}{L^d}. \)
  到中心距离在\(\rho\)以内的结点个数\(\sim \rho^d.\)
\item
  网络中结点的总数

  \begin{align}
  N(t)&=\int_0^{R(t)}\mu(\rho,t)d\rho\\
  &\sim R(t)^{d-1}\\
  &\sim t^{d-1}
  \end{align}
\item
  总边数:

  \begin{itemize}
  \item
    边数\(v(\rho,\Theta,t)\sim u(\rho,\Theta,t)^2\)(局部每两个边都相连)
  \item
    总边数
  \end{itemize}

  \begin{align}
  E(t) &=\int v(\rho,\Theta,t)d\sigma\\
  &\sim \int t^2\cdot t^{d-1} dt\\
  & =t^{d+1}.
  \end{align}

  \begin{itemize}
  \item
    所以我们有\(E(t)\sim N(t)^{\frac{d+2}{d+1}}.\)
  \end{itemize}
\end{itemize}

\begin{itemize}
\item
  标度率

  \begin{enumerate}
  \item
    边数与结点数 \(\gamma=\frac{d+2}{d+1}. \)

    \begin{itemize}
    \item
      证明:总边数:

      边数\(v(\rho,\Theta,t)\sim u(\rho,\Theta,t)^2\)(局部每两个边都相连)

      总边数
    \end{itemize}

    \begin{align}
    E(t) &=\int v(\rho,\Theta,t)d\sigma\\
    &\sim \int t^2\cdot t^{d-1} dt(\text{极坐标变换})\\
    & =t^{d+1}.
    \end{align}

    所以我们有\(E(t)\sim N(t)^{\frac{d+2}{d+1}}\).
  \item
    体积与结点数\(\gamma=\frac{d}{d+1}.\)(这个就不证明了,跟上面差不多)这是一个亚线性的现象(Heap's
    law)。
  \item
    加速增长效应 \(t\sim N(t)^{\frac{1}{d+1}}\).

    \[s.t\ \ \frac{dN(t)}{dt}\sim N{(t)}^{\frac{d}{d+1}}\\
    \text{左边: }N(t)\sim t^{d+1}\\
    dN(t)\sim t^{d}dt\\
    and\ t^d\sim N{(t)}^{\frac{d}{d+1}}\\
    so\ that\frac{dN(t)}{dt}\sim N(t)^{\frac{d}{d+1}}.\]
  \end{enumerate}
\end{itemize}

\subsection{跟真实结果不符合?增加一个参数}

一个新加入的结点的生存概率是插入点的点密度的一个负指数倍:\(P(\text{survive})=\mu(\rho,\Theta,t)^{-\alpha}\).

与之前的推导类似,在某位置\(d\sigma\)附近的结点个数为(从开始有结点开始,对时间积分)

\[\mu(\rho,\Theta,t)d\sigma = \int_{\tau_\rho}^t \mu(\rho,\Theta,s)^{-\alpha} \frac{d\sigma}{L^d}ds\]

我们可以得到一个PDE:$\begin{cases}
\frac{\partial u}{\partial t} =\frac{1}{L^d}\mu^{-\alpha}\\
\mu(\rho,\Theta,\tau_\rho)=0. (\text{初值条件})
\end{cases}$

它的解是\[\mu(\rho,\Theta,t)\sim{(t-\tau_\rho)}^{1/(1+\alpha)}\]
\begin{itemize}
\item
  边:\(R(t)\sim N(t)^{1+\frac{1}{1+(1+\alpha)d}}\)
\item
  体积\(V(t)\sim N(t)^{1+\frac{1+\alpha}{1+(1+\alpha)d}}\)
\end{itemize}
\(\alpha\rightarrow\infty\)时,上述两个幂律指数都会趋近于\(1\).
也就是随着\(\alpha\)的增加,超/亚线性性会降低至线性。