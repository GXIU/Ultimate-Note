\chapter{写作总结}

这是个一言难尽的部分。但是还是要写。形式将是原理结合实例的方式。至于论文绘图部分,我打算整理在\href{https://github.com/GXIU/Easy-Templates}{这个repo}里面。

我对写作的态度非常微妙。一方面我觉得写作不应该成为科研训练的一部分,想得清楚自然写得明白,一个人如果水平足够而且讲得明白,你读不懂他的文章真的是你自己的问题;另一方面看到好的写作,比如Rick Durrett随性俚语的风格、Robert May/Martin Nowak充满热情的描述,我也会感到兴奋。我的观点里,好的科研写作大概应该是\emph{有风格的八股文}。所以这一章我想列出来的是一些自洽逻辑是如何在语言上被组织起来的。

五六年前听王琳琳指挥讲“什么是好的演奏”时,他说了这样的一个观点:“足够的表达力应该都在谱子里写出来了,你只需要‘照做’,其余的东西都在你性格里。”我深以为然。

\section{摘要与引言}

\paragraph{5/27/2020} 带来一首\href{http://dx.doi.org/10.1103/PhysRevLett.112.148103}{Range Expansion of Heterogeneous Populations}~\cite{PhysRevLett.112.148103}。这篇文章特点是一个比较具体的问题:种群是如何在有风险的情况下扩张的。

\textbf{摘要是这样的}:Risk spreading in bacterial populations is generally regarded as a strategy to maximize survival.【主题在领域内的意义;】Here, we study its role during range expansion of a genetically diverse population where growth and motility are two alternative traits.【研究的一个关系,主要变量是什么;】 We find that during the initial expansion phase fast-growing cells do have a selective advantage. 【突出的主要结论;】By contrast, asymptotically, generalists balancing motility and reproduction are evolutionarily most successful. 【一般的主要结论;】These findings are rationalized by a set of coupled Fisher equations complemented by stochastic simulations. 【研究方法:作者是如何导出这些结论的。】

\textbf{Introduction部分}:Expansion of populations is a ubiquitous phenomenon in nature that includes the spreading of advantageous genes [1] or infectious diseases [2,3] and the dispersal of species into new territory. 【我们要研究的这个问题在自然界有哪些自然的对应】 The latter has recently been investigated experimentally by analyzing the spreading of bacterial populations after droplet inoculation on an agar plate [4–10]. 【这些是被如何研究的】 Among others, these studies have highlighted the importance of random genetic drift in driving population differentiation along the expanding fronts of bacterial colonies [7–9,11–13].【这些是被如何研究的】 While these studies have focused on genetically uniform populations or the competition between two strains with different growth rates [12,13], much less is known about the range expansion of heterogeneous populations. 【痛点:不同的特性,种群数量相同的群体的竞争被研究过了,种群数量异质的还有很多未知。】 Single cell studies have revealed that even genetically identical bacteria exhibit variability in phenotypic traits [14]. 【为什么要这么研究呢,因为基因相同,表达也有可能不同(所以要看环境的影响)。】 As an example, clonal populations of Bacillus subtilis (in midexponential growth phase) consist of both swarming cells, propelled by flagella, and nonmotile cells [15]. 【扩张种群里面可能同时包括有鞭毛的和没鞭毛的细胞(表现型不同了)。】 Cells in the motile state do not divide. 【而且,有鞭毛的还不会分化。】 As a result, colonies of B. subtilis are heterogeneous with respect to the cells’ motility. 【所以,对于细胞的运动能力来说,扩张范围是存在异质性的。】 This risk-spreading strategy allows the population to exploit nutrients at its current location and at the same time disperse to new, possibly more favorable, niches. 【这种异质性十分有意义。】

Motivated by these findings, we consider range expansion of a heterogeneous population. 【所以我们要搞这个题目的研究辣!】 We ask what degree of risk spreading between cell division and motility is optimal for survival during range expansion, i.e., whether an individual is better off by investing preferentially in growth or in motility or by adopting a risk-spreading strategy and to balance its investment in growth as well as motility. 【研究问题】 Specifically, we study range expansion dynamics on a one- and two-dimensional lattice, where each site can be occupied by an arbitrary number of individuals. 【数学上的描述】 We assume that each individual $i$ has a distinct genotype $A_i$, which encodes rates to migrate, $e_i$, and reproduce, $\mu_i$; i.e., in the language of game theory each individual plays a mixed strategy. In detail, an individual $A_i$ may reproduce with a rate in the interval $\mu_i\in(0,\mu_{\max})$ upon consumption of resources $B : A_i B \stackrel{\mu_i}{\to}A_iA_i$; the offspring inherits the genotype and is placed on the same lattice site. In addition, individuals are able to migrate upon stochastically hopping to nearest-neighbor sites with a rate $e_i$ in the range $(0, e_{\max})$. Motivated by the behavior of bacterial populations, we assume that an individual may invest its limited resources partly in motility and partly in reproduction and model this by the constraint $e_i / e_{\max} + \mu_i / \mu_{\max} = 1 $; i.e., fast reproducing individuals can only move slowly and vice versa. As we will see, the implications of this biologically motivated trade-off are more intricate than the phenomenon of front acceleration found in populations exhibiting only heterogeneous motility [16]. Numerical simulations of our stochastic lattice gas model were performed using Gillespie’s algorithm [17] with sequential updating on square and hexagonal lattices with lattice spacing $a$. We measure time in units of the inverse maximum reproduction rate $1 / \mu_{\max}$; i.e., roughly speaking dimensionless time $t$ corresponds to the number of generations (of the slowest moving genotype).





% \bibliographystyle{unsrt}
% \bibliography{ref.bib}