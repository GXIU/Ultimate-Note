\chapter{人口动力学}

更新于05-23-2020。

可能SYM论文还是要搞一搞,PRL上面城市相关的内容少之又少,我还是换个思路,搜索了一下人口动力学的内容,结果发现合适的内容很多。先试试。

文献列表:
\begin{itemize}
    \item Evolution Arrests Invasions of Cooperative Populations 
    \item Critical Behaviors in Contagion Dynamics
\end{itemize}

\paragraph{Evolution Arrests Invasions of Cooperative Populations}

从写作开始学吧。

人口扩张是很多生理学和生态学转化的催化剂,举例。尽管人口增多伴随着很多入侵的基因类型,我们证明这并非不能避免的。在合作人口中,减少扩张距离的突变有比较强的竞争优势,所以他们会在扩张的前沿积累,最终组织人口的空间扩张。我们的结果是进化的一个罕见罕见例子:种群可能会向着不希望的方向进行进化。这也会导致新的策略来对抗入侵者。

空间扩张是很多过程的背后原理。空间扩张也是\textbf{唯一}的能使物种种群密度变丰富的过程。很多的入侵因为损害了物种多样性而不被欢迎,不幸的是,控制通常都是无效的,即使控制住了也会反弹。入侵特点的进化和入侵加速的进化是很常见的,比如澳大利亚蟾蜍和人类癌症的进化。\textbf{我们要探讨的是空间扩张的快慢}。

选择快速扩张是有道理的,因为种群可以接触到更多的资源。很多基于非合作人口的理论和实证工作是支持这个观点的:很少的有机体就足够建立一个viable的群体了。但是其他的一些群体,比如肿瘤,是合作地生长的(生态学上,这叫Allee效应)。在这种情况下,如果人口密度低于Allee threshold,种群甚至可能灭绝。

\paragraph{Critical Behaviors in Contagion Dynamics}

本文探讨三种作用下人群中两意见的转换。