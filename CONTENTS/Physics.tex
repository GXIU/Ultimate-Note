\chapter{物理模型}

\section{通过几何重正化对真实网络进行多尺度展开}

\begin{itemize}
\item
  多尺度 与 小世界 的矛盾

  \begin{itemize}
  \item
    欧式长度和对称性的缺陷
  \end{itemize}
\item
  复杂网络的几何重正化群

  \begin{itemize}
  \item
    将真实网络插入到一个度量空间,会体现出geometric scaling特征
  \end{itemize}
\end{itemize}

复杂网络中,多尺度也是共存的,但是它们被一些其他的事限制住了,并不能直接讨论\emph{自相似性}和\emph{标度无关性}。原因是我们没有一种有效的手段来对网络的length
scale进行变换。

\begin{itemize}
\item
  以前的手段:

  \begin{itemize}
  \item
    拓扑/粗粒化/random walks
  \item
    /box-covering/:证明了真实网络有着有限的分形维数,有自相似性
  \item
    拓扑scaling性质只体现在度分布、平均度、最大度上面
  \item
    尽管有很好的度量,最短路径的集合作为研究length-based scaling
    factor还是很不好的数据集。(原因是small-world的存在)
  \end{itemize}
\item
  In this work, we introduce a geometric renormalization group for
  complex networks (RGN). The method is based on a geometric embedding
  of the networks to construct renormalized versions of their structure
  by coase-graining neighbouring nodes into supernodes and defining a
  new map which progressively selects longer range connections by
  identifying relevant interactions at each scale. The RGN technique is
  inspired by the block spin renormal- ization group devised by L. P.
  Kadanoff {[}18{]}.
\end{itemize}

\subsection{真实网络中几何标度存在的证据}

研究对象:复杂网络到hidden度量空间的映射:\(\mathcal M(T,G)\)

  \begin{itemize}
  \item
    定义一个几何重正化算子$\mathbb {F_r}$,得到一个新的拓扑$T'$和一个新的几何图$G'$,由此定义一个新的重正化映射$\mathcal{M}'$: $\mathcal{M}(T,G)\stackrel{\mathbb{F_r}}{\longrightarrow}\mathcal{M}'(T',G')$
  \item
    The transformation zooms out by changing the \textbf{minimum length
    scale} from that of the original network to a larger value.
  \item
    这个过程可以迭代\(O(\ln N)\)次。
  \end{itemize}
例子:

  \begin{itemize}
  \item
    最简单的度量空间:一维圆周:\(\{\theta_i:i=1,2,3,\cdots,N\}\)

    \begin{itemize}
    \item
      重正化步骤:

      \begin{enumerate}
      \def\labelenumi{\arabic{enumi}.}
      \item
        定义block:圆周上挨着的\(r\)个点。
      \item
        粗粒化为超级结点(不管是否连接)每个超级结点都控制一个角区域。所以它们的序关系得以保留。

        \begin{itemize}
        \item
          原连接:

          \begin{itemize}
          \item
            超级结点内
          \item
            超级结点间:建立边
          \end{itemize}
        \end{itemize}
      \end{enumerate}
    \end{itemize}
  \item
    用到的例子:

    \begin{itemize}
    \item
      Internet
    \item
      Airports
    \item
      新陈代谢
    \item
      scripts\ldots\ldots{}
    \end{itemize}
  \end{itemize}
\(\mathbb S_1\)模型:将结点放在一个圆周上,以一定概率分布连接每两个点。两个点越远链接概率越低(similarity),度乘积越大连接概率越高(popularity)。

应用:The RGN enables us to unfold scale-free complex net- works in a
self-similar multi-layer shell which unveils the coexisting scales and
their interplay. Beyond

\begin{itemize}
\item
  Mini-me network replicas.

  \begin{itemize}
  \item
    networked communication systems
  \item
    可以保持微观结构的同时,不破坏介观结构
  \item
    Mini-me replicas can also be used to perform finite size scaling of
    critical phenomena taking place on real networks
  \item
    Typically, the renormalized average degree of real net- works
    increases in the flow, since they belong to the small-world phase
    (see inset in Fig. 3B), meaning that the network layer at the
    selected scale is more densely connected.
  \end{itemize}
\end{itemize}

\section{空间网络的标度律}

想不到这个题目竟然发过PRL,更想不到我今天才看见它(2020/5/25)。